\documentclass[9pt]{article}

\usepackage{amsmath}
\usepackage{tcolorbox}
% `parskip` removes indentation for all paragraphs: http://tex.stackexchange.com/a/55016
\usepackage{parskip}
% Allows us to color rows / cols of a table.
% See https://texblog.org/2011/04/19/highlight-table-rowscolumns-with-color/
\usepackage{color, colortbl}

\usepackage{hyperref}

\leftmargin=0.25in
\oddsidemargin=0.25in
\textwidth=6.0in
\topmargin=-0.25in
\textheight=9.25in

\definecolor{Gray}{gray}{0.9}

\begin{document}

\begin{center}
  \large\textbf{MIT 18.01 Problem Set 4 Unofficial Solutions}
\end{center}

\begin{tcolorbox}
  \textbf{Q1a)} Use the mean value property to show that if $f(0) = 0$ and $f'(x) \geq 0$, then $f(x) \geq 0$ for all $x \geq 0$.
\end{tcolorbox}

For any $x > 0$, $f(x) = f(0) + f'(c)(x - 0)$, for some $0 < c < x$

Since $f'(x) \geq 0$ for all $x$, then for all $x > 0$, $f(x) \geq f(0)$.

Since $f(x) \geq f(0) \geq 0$ for all $x > 0$, then $f(x) \geq 0$ for all $x > 0$.

Since $f(0) = 0$ and $f(x) \geq 0$ for all $x > 0$, then $f(x) \geq 0$ for all $x \geq 0$.


\begin{tcolorbox}
  \textbf{Q1b)} Deduce from part (a) that $ln(1 + x) \leq x$ for $x \geq 0$. Hint: Use $f(x) = x - ln(1 + x)$.
\end{tcolorbox}

Let $f(x) = x - ln(1 + x)$

Then $f(0) = 0 - ln(1 + 0) = 0 - 0 = 0$

$f'(x) = 1 - \frac{1}{1 + x}$

For all $x \geq 0$, $1 + x \geq 1$ and $\frac{1}{1 + x} \leq 1$ and $1 - \frac{1}{1 + x} \geq 0$

Since $f'(x) = 1 - \frac{1}{1 + x} \geq 0$, therefore $f'(x) \geq 0$ for $x \geq 0$.

By 1(a), $f(x) = x - ln(1 + x) \geq 0$ for $x \geq 0$

Then $x \geq ln(1 + x)$ for $x \geq 0$


\begin{tcolorbox}
  \textbf{Q1c)} Use the same method as in (b) to show $ln(1 + x) \geq x - x^2 / 2$ and $ln(1 + x) \leq x - x^2 / 2 + x^3 / 3$ for $x \geq 0$.
\end{tcolorbox}

Let $f_1(x) = ln(1 + x) - x + \frac{x^2}{2}$

$f_1(0) = ln(1 + 0) - 0 + \frac{0^2}{2} = 0$

$f_1'(x) = \frac{1}{1 + x} - 1 + x = \frac{1 - (1 + x) + x(1 + x)}{1 + x} = \frac{1 - 1 - x + x + x^2}{1 + x} = \frac{x^2}{1 + x}$

Since $1 + x \geq 1$ for all $x \geq 0$ and $x^2 \geq 0$ for all $x \geq 0$, then $f_1'(x) = \frac{x^2}{1 + x} \geq 0$ for all $x \geq 0$.

By 1(a), $f_1(x) \geq 0$ for all $x \geq 0$, which is equivalent to $ln(1 + x) - x - \frac{x^2}{2} \geq 0$ for all $x \geq 0$. Hence $ln(1 + x) \geq x - \frac{x^2}{2}$ for all $x \geq 0$.
\\

Let $f_2(x) = x - \frac{x^2}{2} + \frac{x^3}{3} - ln(1 + x)$

$f_2(0) = 0 - \frac{0^2}{2} + \frac{0^3}{3} - ln(1 + 0) = 0$

$f_2'(x) = 1 - x + x^2 - \frac{1}{1 + x} = \frac{1(1 + x) - x(1 + x) + x^2 (1 + x) - 1}{1 + x} = \frac{1 + x - x - x^2 + x^2 + x^3}{1 + x} = \frac{x^3}{1 + x}$

Since $1 + x \geq 1$ for all $x \geq 0$ and $x^3 \geq 0$ for all $x \geq 0$, then $f_2'(x) = \frac{x^3}{1 + x} \geq 0$ for all $x \geq 0$.

By 1(a), $f_2(x) = x - \frac{x^2}{2} + \frac{x^3}{3} - ln(1 + x) \geq 0$ for all $x \geq 0$. Hence $x - \frac{x^2}{2} + \frac{x^3}{3} \geq ln(1 + x)$ for all $x \geq 0$.


\begin{tcolorbox}
  \textbf{Q1d)} Find the pattern in (b) and (c) and make a general conjecture.
\end{tcolorbox}

General conjecture:

$ln(1 + x) \leq x - \frac{x^2}{2} + \frac{x^3}{3} + ... + \frac{x^{2n + 1}}{2n + 1} = \sum\limits_{k=1}^{2n + 1} (-1)^{k+1} \frac{x^k}{k}$ for all $x \geq 0, n \geq 0$

$ln(1 + x) \geq x - \frac{x^2}{2} + \frac{x^3}{3} + ... + \frac{x^{2n}}{2n} = \sum\limits_{k=1}^{2n} (-1)^{k+1} \frac{x^k}{k}$ for all $x \geq 0, n \geq 1$

Let $f_1(x) = (\sum\limits_{k=1}^{2n + 1} (-1)^{k+1} \frac{x^k}{k}) - ln(x + 1)$ for any $n \geq 1$.

$f_1(0) = (\sum\limits_{k=1}^{2n + 1} (-1)^{k+1} \frac{0^k}{k}) - ln(0 + 1) = 0$

\begin{align*}
  f_1'(x) &= (\sum\limits_{k=1}^{2n + 1} (-1)^{k+1} x^{k-1}) - \frac{1}{x + 1} \\
          &= \frac{(x+1)(\sum\limits_{k=1}^{2n + 1} (-1)^{k+1} x^{k-1}) - 1}{x + 1} \\
          &= \frac{(\sum\limits_{k=1}^{2n + 1} (-1)^{k+1} (x^k + x^{k - 1})) - 1}{x + 1} \\
          &= \frac{(x^1 + x^0) - (x^2 + x^1) + ... + (-1)^{2n + 1 + 1}(x^{2n + 1} + x^{2n}) - 1}{x + 1} \\
          &= \frac{x^0 + x^{2n + 1} - 1}{x + 1} \\
          &= \frac{x^{2n + 1}}{x + 1}
\end{align*}

Since $x + 1 \geq 1$ for $x \geq 0$ and $x^{2n + 1} \geq 0$ for $n \geq 0, x \geq 0$, then $f_1'(x) = \frac{x^{2n + 1}}{x + 1} \geq 0$ for all $x \geq 0$.

By 1(a), $f_1(x) = (\sum\limits_{k=1}^{2n+1} (-1)^{k+1} \frac{x^k}{k}) - ln(x + 1) \geq 0$ for all $x \geq 0, n \geq 0$. Hence $ln(x + 1) \leq \sum\limits_{k=1}^{2n+1} (-1)^{k+1} \frac{x^k}{k}$ for all $x \geq 0, n \geq 0$.
\\
\\

Let $f_2(x) = ln(x + 1) - \sum\limits_{k=1}^{2n} (-1)^{k+1} \frac{x^k}{k}$ for any $n \geq 0$.

$f_2(0) = ln(0 + 1) - \sum\limits_{k=1}^{2n} (-1)^{k+1} \frac{0^k}{k} = 0$

\begin{align*}
  f_2'(x) &= \frac{1}{x + 1} - \sum\limits_{k=1}^{2n} (-1)^{k+1} x^{k-1} \\
          &= \frac{1 - (x + 1)\sum\limits_{k=1}^{2n} x^{k - 1}}{x + 1} \\
          &= \frac{1 - \sum\limits_{k=1}^{2n}(x^k + x^{k - 1})}{x + 1} \\
          &= \frac{1 - ((x^1 + x^0) - (x^2 + x^1) + ... + (-1)^{2n + 1}(x^{2n} + x^{2n - 1}))}{x + 1} \\
          &= \frac{1 - (x^0 - x^{2n})}{x + 1} \\
          &= \frac{x^{2n}}{x + 1}
\end{align*}

For $x \geq 0$, $x + 1 \geq 1$ and $x^{2n} \geq 0$ for $x \geq 0, n \geq 1$. Hence $f_2'(x) = \frac{x^{2n}}{x + 1} \geq 0$ for $n \geq 1, x \geq 0$.

By 1(a), $f_2(x) = ln(x + 1) - \sum\limits_{k=1}^{2n} (-1)^{k+1} x^{k-1} \geq 0$ for all $n \geq 1, x \geq 0$.

Then $ln(x + 1) \geq \sum\limits_{k=1}^{2n} (-1)^{k+1} x^{k-1}$ for all $n \geq 1, x \geq 0$.


\begin{tcolorbox}
  \textbf{Q1e)} Show that $ln(1 + x) \leq x$ for $-1 < x \leq 0$. (Use the change of variable $u = -x$.)
\end{tcolorbox}

Let $u = -x$. Then $x = -u$.

We want to show that $ln(1 - u) \leq -u$ for $0 \leq u < 1$.

Let $f(u) = -u - ln(1 - u)$

$f(0) = -0 - ln(1 - 0) = 0$

$f'(u) = -1 - \frac{-1}{1 - u} = -1 + \frac{1}{1-u} = \frac{-1(1-u) + 1}{1 - u} = \frac{u - 1 + 1}{1 - u} = \frac{u}{1 - u}$

Since $0 \leq u < 1$, then $1 - u > 0$. Hence $f'(u) = \frac{u}{1 - u} \geq 0$ for $0 \leq u < 1$

By 1(a), $f(u) = -u - ln(1 - u) \geq 0$ for $0 \leq u < 1$.

Then $f(x) = -(-x) - ln(1 - (-x)) = x - ln(1 + x) \geq 0$ for $0 \leq -x < 1$ or $-1 < x \leq 0$.

Hence $ln(1 + x) \leq x$ for $-1 < x \leq 0$


\begin{tcolorbox}
  \textbf{Q2a)} Do 5.3/68
\end{tcolorbox}

I do not have the textbook. Skipped.


\begin{tcolorbox}
  \textbf{Q2b)} Show that both of the following integrals are correct, and explain.

  \begin{align*}
    \int tan x\ sec^2 x\ dx = (1/2) tan^2 x; \int tan x\ sec^2 x\ dx = (1/2) sec^2 x
  \end{align*}
\end{tcolorbox}

Let $u = tan\ x$. We have

\begin{align*}
  \int tan x\ sec^2 dx &= \int (sec^2 x)\ tan x\ dx \\
                       &= \int u'\ u\ du \\
                       &= \frac{1}{2} u^2 + c \\
                       &= \frac{1}{2} tan^2 x + c
\end{align*}

Now we want to prove that $\int tan x\ sec^2 x\ dx = (1/2) sec^2 x$. Let $u = sec\ x$. Then $\int u' \ u\ du = \frac{1}{2}u^2 + c = \frac{1}{2} sec^2 x + c$

\end{document}
