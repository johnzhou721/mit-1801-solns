\documentclass[9pt]{article}

\usepackage{amsmath}
\usepackage{tcolorbox}
% `parskip` removes indentation for all paragraphs: http://tex.stackexchange.com/a/55016
\usepackage{parskip}
% Allows us to color rows / cols of a table.
% See https://texblog.org/2011/04/19/highlight-table-rowscolumns-with-color/
\usepackage{color, colortbl}

\usepackage{hyperref}
\graphicspath{{images/ps5/}}

\leftmargin=0.25in
\oddsidemargin=0.25in
\textwidth=6.0in
\topmargin=-0.25in
\textheight=9.25in

\definecolor{Gray}{gray}{0.9}

\begin{document}

\begin{center}
  \large\textbf{MIT 18.01 Problem Set 6 Unofficial Solutions}
\end{center}

\begin{tcolorbox}
  \textbf{Q1)} Do 7.4/12 and 13.
\end{tcolorbox}

\textbf{Skipped.} We do not have the textbook.


\begin{tcolorbox}
  \textbf{Q2)} The voltage $V$ of the house current is given by\\
  \begin{center}
    $V(t) = Csin(120\pi t)$
  \end{center}
  where $t$ is time, in seconds and $C$ is a constant amplitude. The square root of the average value of $V^2$ over one period of $V(t)$ (or cycle) is called the \emph{root-mean-square} voltage, abbreviated RMS. This is what the voltage meter on a house records. For house current, find the RMS in terms of the constant $C$. (The peak voltage delivered to the house is $\pm C$. The units of $V^2$ are square volts; when we take the square root again after averaging, the units become volts again.)
\end{tcolorbox}

Every cycle of the $sin$ function corresponds to $2 \pi$. This happens every $t = \frac{2 \pi}{120 \pi} = \frac{1}{60}$ seconds.

Since $V(t) = Csin(120\pi t)$, $V^2(t) = C^2 sin^2(120\pi t)$. The average value of $V^2$ over 1 cycle of $V(t)$ is:

\begin{align*}
  \frac{1}{\frac{1}{60} - 0} \int_0^{\frac{1}{60}} C^2 sin^2(120\pi t) dt &= 60C^2 \int_0^{\frac{1}{60}} sin^2(120\pi t) dt \\
  &= 60C^2(\frac{1}{2}t - \frac{1}{240\pi} sin(120\pi t) cos(120\pi t)) \bigg]_0^{1/60} \\
  &= 60C^2(\frac{1}{2} \cdot \frac{1}{60} - \frac{1}{240\pi} sin(2\pi) cos(2\pi) - (\frac{1}{2} \cdot 0 - \frac{1}{240\pi} sin(0) cos(0))) \\
  &= 60C^2 \cdot \frac{1}{120} \\
  &= \frac{C^2}{2}
\end{align*}

Then RMS $= \sqrt{\frac{C^2}{2}} = \frac{C}{\sqrt{2}} = \frac{\sqrt{2}C}{2}$

\end{document}
